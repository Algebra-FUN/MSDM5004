\documentclass[runningheads]{llncs}

\usepackage{graphicx}
\usepackage{subcaption}
\captionsetup{compatibility=false}
\usepackage{amsmath}
\usepackage{amssymb}
\usepackage[colorlinks=true, linkcolor=blue]{hyperref}
\usepackage{tabularx}
\newcolumntype{C}{>{\centering\arraybackslash}X}
\usepackage{listings}
\renewcommand\lstlistingname{Code}
\usepackage[usenames,dvipsnames]{xcolor}

\lstdefinelanguage{Julia}
  {morekeywords={abstract,break,case,catch,const,continue,do,else,elseif,%
      end,export,false,for,function,immutable,import,importall,if,in,%
      macro,module,otherwise,quote,return,switch,true,try,type,typealias,%
      using,while},
   sensitive=true,
   alsoother={\$},
   morecomment=[l]\#,
   morecomment=[n]{\#=}{=\#},
   morestring=[s]{"}{"},
   morestring=[m]{'}{'},
}[keywords,comments,strings]

\lstset{%
    language         = Julia,
    basicstyle       = \ttfamily,
    keywordstyle     = \bfseries\color{blue},
    stringstyle      = \color{magenta},
    commentstyle     = \color{ForestGreen},
    showstringspaces = false,
}

\definecolor{codegreen}{rgb}{0,0.6,0}
\definecolor{codegray}{rgb}{0.5,0.5,0.5}
\definecolor{codepurple}{rgb}{0.58,0,0.82}
\definecolor{backcolour}{rgb}{0.95,0.95,0.92}

\lstdefinestyle{mystyle}{
    backgroundcolor=\color{backcolour},   
    commentstyle=\color{codegreen},
    keywordstyle=\color{magenta},
    numberstyle=\tiny\color{codegray},
    stringstyle=\color{codepurple},
    basicstyle=\ttfamily\footnotesize,
    breakatwhitespace=false,         
    breaklines=true,                 
    captionpos=b,                    
    keepspaces=true,                 
    numbers=left,                    
    numbersep=5pt,                  
    showspaces=false,                
    showstringspaces=false,
    showtabs=false,                  
    tabsize=2
}
\lstset{style=mystyle}

\begin{document}

\title{
    MSDM5004 Project I\\
    Evolution of Surface of a Stressed Solid}
\titlerunning{MSDM5004 Project I}
\author{FAN Yifei}
\authorrunning{Y. FAN}
\institute{HKUST, Hong Kong SAR, China\\
    \email{\href{mailto:yfanba@ust.hk}{yfanba@ust.hk}}}
\maketitle

\begin{abstract}
    In this project, we use FFT method to solve the PDE for the evolution of the surface of a stressed solid. 
    We carry out numerical simulation and get result. 
    Then we analyze the elementary reason why there are different result for different $r$.\\
    The code of this project is available at \url{https://github.com/Algebra-FUN/MSDM5004/tree/main/Project/1}.
    \keywords{PDE \and FFT method\and Evolution of Surface.}
\end{abstract}

\section{Introduction for PDE of Evolution of Surface}

The surface of a solid under stress is unstable to small perturbations. 
Assume the height of the solid surface is $h(x, t)$, where $x$ is the spatial variable and $t$ is the time. 
The evolution of the surface is described by the PDE in Eq.~\eqref{eq:1}.

\begin{equation}
    \label{eq:1}
    h_t = r^2\frac{\partial}{\partial t}\left\{ -4rH(h_x)-r^2h_{xx}+4r^2h_x^2+8r^2hh_{xx}+4r^2H^2(h_x)+8r^2H(hH(h_x))_x\right\},
\end{equation}
where r > 0 is a constant, and $H$ is the Hilbert transform.

\section{FFT Method}

The Fourier transform of $H(g)(x)$ is

\begin{equation}
    \widehat{H(g)}(k)=-i\mbox{sgn}(k)\hat{g}(k).
\end{equation}
Thus, we have $\widehat{H(h_x)}(k)=|k|\hat{h}(k)$.

We use FFT method (in fact, pseudo-spectral method) to solve this PDE. That is, we evolve the PDE in the Fourier space:

\begin{equation}
    \label{eq:fft_pde}
    \frac{\partial}{\partial t}\hat{h}=\lambda_k\hat{h}-r^2k^2\hat{f},
\end{equation}
where $\hat{h}$ is the Fourier transform of $h$,

\begin{equation}
    \label{eq:2}
    f=r^2[4h_x^2+8hh_{xx}+4H^2(h_x)+8H(hH(h_x))_x],
\end{equation}
and $\hat{f}$ is the Fourier transform of $f$, and 

\begin{equation}
    \lambda_k=4r^3|k|^3-r^4k^4.
\end{equation}

In order to solve this PDE in frequency space, we need to compute the Hilbert transform of $f$ in Eq.~\eqref{eq:2}.
Using the basic properties of differentiation and the convolution theorem of the Fourier transform, we have

\begin{equation}
    \label{eq:hatf}
    \hat{f}=r^2[12 (|k| \hat{h})*(|k| \hat{h})-8 (k^2 \hat{h})*\hat{h}-4(k \hat{h})*(k \hat{h})],
\end{equation}
where $*$ denote the operation of convolution.

We use trapezoidal rule for the time discretization for the linear part and compute the nonlinear part explicitly, the numerical scheme is

\begin{equation}
    \label{eq:3}
    \frac{\hat{h}^{n+1}_k-\hat{h}^{n}_k}{\Delta t}=\lambda_k\frac{\hat{h}^{n+1}_k+\hat{h}^{n}_k}2-r^2k^2\left(\frac32\hat{f}^n_k-\frac12\hat{f}^{n-1}_k\right),
\end{equation}
where $\Delta t$ is the time step, and $\hat{f}^{-1}_k=\hat{f}^{0}_k$. Here the notation $\hat{h}^n_k=\hat{h}(k,t_n)$ with $t_n=n\Delta t$.

Using this Eq.~\eqref{eq:3}, we can compute the $\hat{h}$ by iteration.

\section{Numerical Simulation}

\subsubsection{Requirements}

Consider the initial condition $h(x,0) = 0.01\cos x$ for periodic domain $x \in [0,2\pi]$, which is the planar surface $h = 0$ with a small perturbation. 
We simulate evolution of the surface $h$ for the following setting of $r$:


\begin{enumerate}
    \item $r = 1.5$,
    \item $r = 3.8$,
    \item $r = 5$.
\end{enumerate}

\subsubsection{Hyperparameters}

The configuration of hyperparameters is shown in Table~\ref{tab:1}.

\begin{table}[htbp]
    \centering
    \caption{Configuration of hyperparameters for the simulation}
    \label{tab:1}
    \begin{tabularx}{\textwidth}{|C|C|C|}
        \hline
        notation & description & value \\
        \hline
        $N$   & number of nodes of the planar surface & 101   \\
        $M$   & total number of time step & 40   \\
        $\Delta t$   & time step & 0.01(0.002 for $r=5$)   \\
        \hline
    \end{tabularx}
\end{table}

\subsubsection{Implementation}

We implement the numerical simulation in Julia.

\begin{enumerate}
\item We find that the input error from the fft function(from FFTW.jl library\cite{FFTW}) would be amplified by the iteration of Eq.~\eqref{eq:2},
which will lead to the instability of the numerical scheme.
In order to solve this issue, we calculate the Fouries transform of $h_0=0.01\cos x$ by hand to get the initial value of $\hat{h}$ avoiding the input error from the fft function.

\item There is a gap between theorem and code. In theoretical analysis, the FFT normalizes on forward step, but in the code, the FFT normalizes on backward step. 
Meanwhile, in this theoretical analysis, we use $k$ as the angle frequency, but in the code, we use $k$ as the times frequency. So we should write a wrapper(Code \ref{FFTWrapper}) to coordinate the difference.

\end{enumerate}

\section{Results and Discussion}

\subsubsection{Simulation Results}

The numerical results are shown in Fig.~\ref{fig:1}.

\begin{figure}[!htbp]
    \centering
    \begin{subfigure}{.4\textwidth}
        \centering
        \includegraphics[width=\linewidth]{../img/surface_snapshot(r=1.5,M=40).pdf}  
    \end{subfigure}
    \begin{subfigure}{.4\textwidth}
        \centering
        \includegraphics[width=\linewidth]{../img/surface_snapshot(r=3.8,M=40).pdf}  
    \end{subfigure}
    \begin{subfigure}{.4\textwidth}
        \centering
        \includegraphics[width=\linewidth]{../img/surface_snapshot(r=5.0,M=40).pdf}  
    \end{subfigure}
    \caption{Surface profiles $h$ when $M=40$}
    \label{fig:1}
\end{figure}
This result seems reasonable. 
And we also compare it with the result in the paper\cite{dong2023corrosion}, the tends of surface morphology evolution process is similar.

For clearly showing the evolution of the surface, we also plot the 3d surface plot in Fig.~\ref{fig:2}.

\begin{figure}[!htbp]
    \centering
    \begin{subfigure}{.4\textwidth}
        \centering
        \includegraphics[width=\linewidth]{../img/surface_3d(r=1.5,M=40).pdf}  
    \end{subfigure}
    \begin{subfigure}{.4\textwidth}
        \centering
        \includegraphics[width=\linewidth]{../img/surface_3d(r=3.8,M=40).pdf}  
    \end{subfigure}
    \begin{subfigure}{.4\textwidth}
        \centering
        \includegraphics[width=\linewidth]{../img/surface_3d(r=5.0,M=40).pdf}  
    \end{subfigure}
    \caption{3d surface plot of $h$ when $M=40$.}
    \label{fig:2}
\end{figure}

You may think the total time is too short since the $M$ is only $40$, 
we also simulate the evolution of the surface $h$ for lone time by setting $M=1000$, the results are shown in Fig.~\ref{fig:3}.

\begin{figure}[!htbp]
    \centering
    \begin{subfigure}{.4\textwidth}
        \centering
        \includegraphics[width=\linewidth]{../img/surface_snapshot(r=1.5,M=1000).pdf}  
    \end{subfigure}
    \begin{subfigure}{.4\textwidth}
        \centering
        \includegraphics[width=\linewidth]{../img/surface_snapshot(r=3.8,M=1000).pdf}  
    \end{subfigure}
    \caption{Surface profiles $h$ when $M=1000$}
    \label{fig:3}
\end{figure}
From the Fig.~\ref{fig:3}, for the case $r=1.5,3.8$, we can see that only the last line is obvious 
and other lines are rendered like horizon lines since their value are much less than the last one which means the $h$ diverge with time exponentially.
So there is no need to simulate the evolution of the surface $h$ for lone time for the case $r=1.5,3.8$ until numerical overflow.

\subsubsection{Discussion}
From the simulation results in Fig.~\ref{fig:1} and Fig.~\ref{fig:2}, we can see that the planar surface $h$ is unstable for $r=1.5,3.8$, but stable for $r=5.0$.

The planar surface tends to stretch for $r=1.5,3.8$, and the initial perturbation is amplified with time.
The effect of the stretching is more obvious for $r=3.8$ than $r=1.5$.

While, the planar surface tends to converge and be stable for $r=5.0$, and the initial perturbation is damped with time.
And we can see the planar surface $h$ becomes $0$ after a long time.

By calculating the $\lambda_k$, the simplified Eq.~\eqref{eq:4} is shown as follows:

\begin{equation}
    \label{eq:4}
    \lambda_k = (4-rk)r^3k^3
\end{equation}
where $k \leq 1$, since we use rfft and all the $k$ is positive integer(without considering the zero frequency in this case).

Then when $r > 4$, $\lambda_k < 0$. 
In the Eq.~\eqref{eq:fft_pde}, $\frac{\partial}{\partial t}\hat{h}$ will negatively response to $\hat{h}$, which means the perturbation will be damped with time.

\section{Conclusion}

In conclusion, we use FFT method to solve the PDE for the evolution of the surface of a stressed solid. 
We use Julia to implement the code for numerical solution and overcome the gap between the theorem and code.
Then we simulate the evolution of the surface for different $r$ and find that the planar surface is unstable for $r=1.5,3.8$, but stable for $r=5.0$.
Finally, we analyze the elementary reason for the stability and instability of the planar surface.



\bibliographystyle{splncs04}
\bibliography{refs}

\newpage
\section*{Appendix}
\subsection*{Code}

The code of this project is available at \url{https://github.com/Algebra-FUN/
MSDM5004/tree/main/Project/1}. You are recommended to view the code online, since the listings package can't render unicode in code well.

\lstinputlisting[caption={FFTW Wrapper}, captionpos=t, label={FFTWrapper}, language=Julia]{../code/FFTWrapper.jl}

\lstinputlisting[caption={ES PDE Solver}, captionpos=t, label={ESPDESolve}, language=Julia]{../code/ESPDESolve.jl}

\end{document}

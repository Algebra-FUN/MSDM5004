\documentclass[runningheads]{llncs}

\usepackage{graphicx}
\usepackage{subcaption}
\captionsetup{compatibility=false}
\usepackage{amsmath}
\usepackage{amssymb}
\usepackage[colorlinks=true, linkcolor=blue]{hyperref}
\usepackage{tabularx}
\newcolumntype{C}{>{\centering\arraybackslash}X}
\usepackage{listings}
\usepackage[usenames,dvipsnames]{xcolor}

\definecolor{backcolour}{rgb}{0.95,0.95,0.92}
\definecolor{codegreen}{rgb}{0,0.6,0}
\definecolor{codepurple}{rgb}{0.58,0,0.82}
\lstdefinestyle{pyStyle}{
    backgroundcolor=\color{backcolour},   
    commentstyle=\color{codegreen},
    keywordstyle=\color{magenta},
    stringstyle=\color{codepurple},
    basicstyle=\ttfamily\footnotesize,
    breakatwhitespace=false,         
    breaklines=true,                 
    keepspaces=true,                 
    numbers=left,       
    numbersep=5pt,                  
    showspaces=false,                
    showstringspaces=false,
    showtabs=false,                  
    tabsize=2,
}
\lstset{style=pyStyle, language=Python}

\begin{document}

\title{
    MSDM5004 Project II\\
    Calculation of a Turbulence Metric by Spectral Methods on Wind Data}
\titlerunning{MSDM5004 Project II}
\author{FAN Yifei}
\authorrunning{Y. FAN}
\institute{HKUST, Hong Kong SAR, China\\
    \email{\href{mailto:yfanba@ust.hk}{yfanba@ust.hk}}}
\maketitle

\begin{abstract}
    In this project, we will perform spectral analysis on one QAR dataset to compute the EDR using a wind-based algorithm, 
    and compare the results with that provided by a commercial software employing the NRL algorithm. 
    Then We will apply the same methods on an AIMMS20 dataset to calculate the EDRs.\\
    The code of this project is available at \url{https://github.com/Algebra-FUN/MSDM5004/tree/main/Project/2}.
    \keywords{Spectral Analysis \and Turbulence \and EDR}
\end{abstract}

\section{Introduction}

Atmospheric turbulence is a weather phenomenon which is a main cause of reported weather-related aircraft incidents and accidents. The impact of turbulence depends on magnitude and size of eddies in the air as well as other aircraft type specific parameters.
Among the wide range of eddy sizes in the atmosphere, aircraft mainly respond to eddies having sizes in the order of tens to hundreds of meters. Eddies of this size can be created by an energy cascade, i.e., a larger eddy in the atmosphere breaking up and forming a smaller eddy, which eventually dissipating to heat. Because this size range is outside the current scope of state-of-the-art operational Numerical Weather Prediction models, these turbulence forecasts are often still diagnostic based through parameterization, i.e., parameters used in algorithms producing these forecasts often need to be “calibrated” using turbulence observations.

One standard metric to quantify turbulence is the “cubic root of eddy dissipation rate” (EDR). 
EDR can be estimated by wind-based algorithm or acceleration-based algorithm, the current project focus on the wind-based method.

In this project, we will perform spectral analysis on one QAR dataset to compute the EDR using a wind-based algorithm, and compare the results with that provided by a commercial software employing the NRL algorithm. 
We will then apply the same methods on an AIMMS20 dataset to calculate the EDRs.

\section{Turbulence Theory}

According to turbulence theory, frequency spectra of transverse velocity (vertical wind component being one) in the inertial sub-range of turbulence regime is given by:

\begin{equation}
    S_{\perp}(f)=C_k^{'}\left(\frac{U}{2\pi}\right)^{\frac23}\epsilon^{\frac23}f^{-\frac53},
\end{equation}

where $f$ is the frequency; $U$ is the average true air speed within the sampling window; $\epsilon$ the dissipation and $C^{'}_k$ being the Kolmogorov constant for transverse
component which is around $0.65$. From this equation, we can find the relation between power spectrum of the vertical wind component and the frequency follows a $-\frac53$ power law.

\section{Data Exploration and Verification(Task 1)}

In this section, we will explore the QAR dataset and verify the $-\frac53$ power law by fitting a linear relation to the log-log plot of the power spectrum of the vertical wind component in QAR dataset.

First, we load the QAR dataset and calculate the fourier transform of the vertical wind component. 
Then we calculate the power spectrum of the vertical wind component 
and plot the log-log plot of the power spectrum of the vertical wind component in Fig.~\ref{fig:qar_loglog},
where we can see the $\ln{S}$ is proportional to $\ln{f}$ and the line is quite noisy.

\begin{figure}[htbp]
    \centering
    \includegraphics[width=\textwidth]{../img/qar_loglog.pdf}
    \caption{Log-log plot of the power spectrum of the vertical wind component in QAR dataset}
    \label{fig:qar_loglog}
\end{figure}

In order to identify the frequency range, bounded by $\omega_1$ and $\omega_2$ that best demonstrates the $-\frac53$ power law,
we should experiment with different sampling window on the vertical wind component to smooth the line. 
We experiment with different sampling window, varying from 10 second to 2 minutes. The result is shown in Fig.~\ref{fig:1}.
We show the fitted line coefficient and fitted goodness metric $R^2$ in the plot.

\begin{figure}
    \centering
    \begin{subfigure}{.4\textwidth}
        \centering
        \includegraphics[width=\linewidth]{../img/qar_loglog(w=10).pdf}  
    \end{subfigure}
    \begin{subfigure}{.4\textwidth}
        \centering
        \includegraphics[width=\linewidth]{../img/qar_loglog(w=20).pdf}  
    \end{subfigure}
    \begin{subfigure}{.4\textwidth}
        \centering
        \includegraphics[width=\linewidth]{../img/qar_loglog(w=50).pdf}  
    \end{subfigure}
    \begin{subfigure}{.4\textwidth}
        \centering
        \includegraphics[width=\linewidth]{../img/qar_loglog(w=120).pdf}  
    \end{subfigure}
    \caption{Log-log plot of the power spectrum of the vertical wind component in QAR dataset with different sampling window}
    \label{fig:1}
\end{figure}

From Fig.~\ref{fig:1}, the line between $0.1$ and $1$ is the smoothest and most straight part in the curve.
Then we choose $\omega_1=0.1$ and $\omega_2=1$ in following calculation.

\section{Calculation of EDR for QAR Dataset}

In this section, we will calculate the EDR for QAR dataset using MLE method and NLR method.

\subsection{ML Method(Task 2)}

The Eddy Dissipation Rate (EDR) can be estimated by the Maximum Likelihood method with the following equation:

\begin{equation}
    \mbox{EDR}=\epsilon^{\frac13}=\left(\frac{2\pi}{U}\right)^{\frac13}\left[\frac1N \sum_{f=\omega_1}^{\omega_2}\left(\frac{S(f)f^{\frac53}}{C^{'}_k}\right)\right]^{\frac12},
\end{equation}

where $\omega_1$ and $\omega_2$ is the cut-off frequency, $S(f)$ is the power spectrum of the vertical wind component, $U$ is the average true air speed within the sampling window, $N$ is the number of data points in the sampling window, and $C^{'}_k$ is the Kolmogorov constant for transverse component which is around $0.65$.

\paragraph{Implementation}

We implement the ML method in Python. There is some details in the implementation:

\begin{itemize}
    \item The fft function in Numpy\cite{fft} is different from the theoretical definition of Fourier transform, the fft function in numpy normalizes on backward transform defaultly. 
    So we should set the parameter "norm" to "forward" to get the right spectrum.
    \item We also can use rfft and irfft in numpy to calculate the Fourier transform and inverse Fourier transform of real sequence, which is faster than fft and ifft.
\end{itemize}

\paragraph{Experiment}

We carry out experiment with different sampling window $w \in [5,10,20]$, the result is shown in Fig.~\ref{fig:2}.

\begin{figure}
    \centering
    \begin{subfigure}{.4\textwidth}
        \centering
        \includegraphics[width=\linewidth]{../img/qar_edr(w=5).pdf}
    \end{subfigure}
    \begin{subfigure}{.4\textwidth}
        \centering
        \includegraphics[width=\linewidth]{../img/qar_edr(w=10).pdf}
    \end{subfigure}
    \begin{subfigure}{.4\textwidth}
        \centering
        \includegraphics[width=\linewidth]{../img/qar_edr(w=20).pdf}
    \end{subfigure}
    \caption{Estimation EDR of QAR dataset using ML method with different sampling window}
    \label{fig:2}
\end{figure}

From Fig.~\ref{fig:2}, when the sampling window $w=5$, the estimated EDR $\hat{\epsilon}^{\frac13}$ achieves the best MSE loss.

\subsection{NLR Method(Task 3)}

The Eddy Dissipation Rate (EDR) can be estimated by the NLR algorithm with the following equation:

\begin{equation}
    \epsilon^{\frac13}=\frac{\widehat{\sigma_w}}{\sqrt{1.05V_a^{\frac23}\left(\omega_1^{-\frac23}-\omega_2^{-\frac23}\right)}}
\end{equation}

where the $\widehat{\sigma_w}$ is the running standard deviation of the band-pass-filtered vertical wind component with cut-off frequency $\omega_1$ and $\omega_2$, 
$V_a$ is the low-pass-filtered airspeed. In this experiment, we use the airspeed within sampling window as $V_a$.

\paragraph{Experiment}

We carry out experiment with different sampling window $w \in [5,10,20]$, the result is shown in Fig.~\ref{fig:3}.

\begin{figure}
    \centering
    \begin{subfigure}{.4\textwidth}
        \centering
        \includegraphics[width=\linewidth]{../img/qar_edr_nlr(w=5).pdf}
    \end{subfigure}
    \begin{subfigure}{.4\textwidth}
        \centering
        \includegraphics[width=\linewidth]{../img/qar_edr_nlr(w=10).pdf}
    \end{subfigure}
    \begin{subfigure}{.4\textwidth}
        \centering
        \includegraphics[width=\linewidth]{../img/qar_edr_nlr(w=20).pdf}
    \end{subfigure}
    \caption{Estimation EDR of QAR dataset using NLR algorithm with different sampling window}
    \label{fig:3}
\end{figure}

From Fig.~\ref{fig:3}, when the sampling window $w=10$, the estimated EDR $\hat{\epsilon}^{\frac13}$ achieves the best MSE loss.

\section{Calculation of EDR for AIMMS20 Dataset(Task 4)}

In this section, we will calculate the EDR for AIMMS20 dataset using ML method and NLR method.

First of all, we load the data and plot the power spectrum of the vertical wind component in Fig.~\ref{fig:4}.

\begin{figure}
    \centering
    \includegraphics[width=.6\linewidth]{../img/probe_loglog.pdf}
    \caption{Log-log plot of the power spectrum of the vertical wind component in AIMMS20 dataset}
    \label{fig:4}
\end{figure}

We choose $\omega_1=0.1$ and $\omega_2=1$ in following calculation.

First, we calculate the EDR using ML method and NLR algorithm with different sampling window $w=5,10,20$ and vertical velocity, the result is shown in Fig.~\ref{fig:6}.

\begin{figure}
    \centering
    \begin{subfigure}{.4\textwidth}
        \centering
        \includegraphics[width=\linewidth]{../img/probe_edr(w=5).pdf}
    \end{subfigure}
    \begin{subfigure}{.4\textwidth}
        \centering
        \includegraphics[width=\linewidth]{../img/probe_edr(w=10).pdf}
    \end{subfigure}
    \begin{subfigure}{.4\textwidth}
        \centering
        \includegraphics[width=\linewidth]{../img/probe_edr(w=20).pdf}
    \end{subfigure}
    \caption{Estimation EDR of AIMMS20 dataset using ML method with different sampling window and vertical velocity}
    \label{fig:6}
\end{figure}

% Furthemore, in order to get a more precise estimation of EDR, we calculate transverse velocity instead of only vertical velocity.

% The transverse velocity is calculated by the following equation:

% \begin{equation}
%     w_{\perp}=\sqrt{|\vec{w}|^2-\left(\frac{\vec{w}\cdot \vec{v}}{|\vec{v}|}\right)}
% \end{equation}

% where $v\in\mathbb{R}$ is the aircraft velocity vector and $w\in\mathbb{R}$ is the wind velocity vector.

% We compute EDR using ML method and NLR algorithm with different sampling window $w=5,10,20$ and transverse velocity, the result is shown in Fig.~\ref{fig:5}.

% \begin{figure}
%     \centering
%     \begin{subfigure}{.4\textwidth}
%         \centering
%         \includegraphics[width=\linewidth]{../img/probe_edr_t(w=5).pdf}
%     \end{subfigure}
%     \begin{subfigure}{.4\textwidth}
%         \centering
%         \includegraphics[width=\linewidth]{../img/probe_edr_t(w=10).pdf}
%     \end{subfigure}
%     \begin{subfigure}{.4\textwidth}
%         \centering
%         \includegraphics[width=\linewidth]{../img/probe_edr_t(w=20).pdf}
%     \end{subfigure}
%     \caption{Estimation EDR of AIMMS20 dataset using ML method with different sampling window}
%     \label{fig:5}
% \end{figure}

\section{Conclusion}

% \section{PDE for Evolution of Surface}

% The surface of a solid under stress is unstable to small perturbations. 
% Assume the height of the solid surface is $h(x, t)$, where $x$ is the spatial variable and $t$ is the time. 
% The evolution of the surface is described by the PDE in Eq.~\eqref{eq:1}.

% \begin{equation}
%     \label{eq:1}
%     h_t = r^2\frac{\partial}{\partial t}\left\{ -4rH(h_x)-r^2h_{xx}+4r^2h_x^2+8r^2hh_{xx}+4r^2H^2(h_x)+8r^2H(hH(h_x))_x\right\},
% \end{equation}
% where r > 0 is a constant, and $H$ is the Hilbert transform.

% \section{FFT Method}

% The Fourier transform of $H(g)(x)$ is

% \begin{equation}
%     \widehat{H(g)}(k)=-i\mbox{sgn}(k)\hat{g}(k).
% \end{equation}
% Thus, we have $\widehat{H(h_x)}(k)=|k|\hat{h}(k)$.

% We use FFT method (in fact, pseudo-spectral method) to solve this PDE. That is, we evolve the PDE in the Fourier space:

% \begin{equation}
%     \label{eq:fft_pde}
%     \frac{\partial}{\partial t}\hat{h}=\lambda_k\hat{h}-r^2k^2\hat{f},
% \end{equation}

% where $\hat{h}$ is the Fourier transform of $h$,

% \begin{equation}
%     \label{eq:2}
%     f=r^2[4h_x^2+8hh_{xx}+4H^2(h_x)+8H(hH(h_x))_x],
% \end{equation}

% and $\hat{f}$ is the Fourier transform of $f$, and 

% \begin{equation}
%     \lambda_k=4r^3|k|^3-r^4k^4.
% \end{equation}

% In order to solve this PDE in frequency space, we need to compute the Hilbert transform of $f$ in Eq.~\eqref{eq:2}.
% Using the basic properties of differentiation and the convolution theorem of the Fourier transform, we have

% \begin{equation}
%     \hat{f}=r^2[12 (|k| \hat{h})*(|k| \hat{h})-8 (k^2 \hat{h})*\hat{h}-4(k \hat{h})*(k \hat{h})],
% \end{equation}

% where $*$ denote the opearation of convolution.

% We use trapezoidal rule for the time discretization for the linear part and compute the nonlinear part explicitly, the numerical scheme is

% \begin{equation}
%     \label{eq:3}
%     \frac{\hat{h}^{n+1}_k-\hat{h}^{n}_k}{\Delta t}=\lambda_k\frac{\hat{h}^{n+1}_k+\hat{h}^{n}_k}2-r^2k^2\left(\frac32\hat{f}^n_k-\frac12\hat{f}^{n-1}_k\right),
% \end{equation}

% where $\Delta t$ is the time step, and $\hat{f}^{-1}_k=\hat{f}^{0}_k$. Here the notation $\hat{h}^n_k=\hat{h}(k,t_n)$ with $t_n=n\Delta t$.

% Using this Eq.~\eqref{eq:3}, we can compute the $\hat{h}$ by iteration.

% \section{Numerical Simulation}

% \subsubsection{Requirements}

% Consider the initial condition $h(x,0) = 0.01\cos x$ for periodic domain $x \in [0,2\pi]$, which is the planar surface $h = 0$ with a small perturbation. 
% We simulate evolution of the surface $h$ for the following setting of $r$:


% \begin{enumerate}
%     \item $r = 1.5$,
%     \item $r = 3.8$,
%     \item $r = 5$.
% \end{enumerate}

% \subsubsection{Hyperparameters}

% The configuration of hyperparameters is shown in Table~\ref{tab:1}.

% \begin{table}[htbp]
%     \centering
%     \caption{Configuration of hyperparameters for the simulation}
%     \label{tab:1}
%     \begin{tabularx}{\textwidth}{|C|C|C|}
%         \hline
%         notation & description & value \\
%         \hline
%         $N$   & number of nodes of the planar surface & 100   \\
%         $M$   & total time step & 40   \\
%         $\Delta t$   & time step & 0.01(0.002 for $r=5$)   \\
%         \hline
%     \end{tabularx}
% \end{table}

% \subsubsection{Implementation}

% We implement the numerical simulation in Julia.

% \begin{enumerate}
% \item We find that the input error from the fft function(from FFTW.jl library\cite{FFTW}) would be amplified by the iteration of Eq.~\eqref{eq:2},
% which will lead to the instability of the numerical scheme.
% In order to solve this issue, we calculate the Fouries transform of $h_0=0.01\cos x$ by hand to get the initial value of $\hat{h}$ avoiding the input error from the fft function.

% \item There is a gap between theorem and code. In theroetical analysis, the FFT normalizes on forward step, but in the code, the FFT normalizes on backward step. 
% Meanhile, in this theroetical analysis, we use $k$ as the angle frequency, but in the code, we use $k$ as the time frequency. So so we should write a wrapper to coordinate the difference.

% \item We use rfft and irfft to accelerate the computation, since the input is real number.
% \end{enumerate}

% \section{Results and Discussion}

% \subsubsection{Simulation Results}

% The numerical results are shown in Fig.~\ref{fig:1}.

% \begin{figure}
%     \centering
%     \begin{subfigure}{.4\textwidth}
%         \centering
%         \includegraphics[width=\linewidth]{../img/surface_snapshot(r=1.5).pdf}  
%     \end{subfigure}
%     \begin{subfigure}{.4\textwidth}
%         \centering
%         \includegraphics[width=\linewidth]{../img/surface_snapshot(r=3.8).pdf}  
%     \end{subfigure}
%     \begin{subfigure}{.4\textwidth}
%         \centering
%         \includegraphics[width=\linewidth]{../img/surface_snapshot(r=5.0).pdf}  
%     \end{subfigure}
%     \caption{Surface profiles $h$ at different time for $r=1.5,3.8,5.0$}
%     \label{fig:1}
% \end{figure}

% This result seems reasonable. 
% And we also compare it with the result in the paper\cite{dong2023corrosion}, the tends of surface morphology evolution process is similar.

% For clearly showing the evolution of the surface, we also plot the 3d surface plot in Fig.~\ref{fig:2}.

% \begin{figure}
%     \centering
%     \begin{subfigure}{.4\textwidth}
%         \centering
%         \includegraphics[width=\linewidth]{../img/surface_3d(r=1.5).pdf}  
%     \end{subfigure}
%     \begin{subfigure}{.4\textwidth}
%         \centering
%         \includegraphics[width=\linewidth]{../img/surface_3d(r=3.8).pdf}  
%     \end{subfigure}
%     \begin{subfigure}{.4\textwidth}
%         \centering
%         \includegraphics[width=\linewidth]{../img/surface_3d(r=5.0).pdf}  
%     \end{subfigure}
%     \caption{3d surface plot of $h$ at different time for $r=1.5,3.8,5.0$}
%     \label{fig:2}
% \end{figure}

% \subsubsection{Discussion}

% From the simulation results, we can see that the planar surface $h$ is unstable for $r=1.5,3.8$, but stable for $r=5.0$.

% The planar surface tends to stretch for $r=1.5,3.8$, and the initial perturbation is amplified with time.
% The effect of the stretching is more obvious for $r=3.8$ than $r=1.5$.

% While, the planar surface tends to converge and be stable for $r=5.0$, and the initial perturbation is damped with time.
% And we can see the planar surface $h$ becomes $0$ after a long time.

% By calculating the $\lambda_k$, the simplified Eq.~\eqref{eq:4} is shown as follows:

% \begin{equation}
%     \label{eq:4}
%     \lambda_k = (4-rk)r^3k^3
% \end{equation}

% where $k \leq 1$, since we use rfft and all the $k$ is positive integer(without considering the zero frequency in this case).

% Then when $r > 4$, $\lambda_k < 0$. 
% In the Eq.~\eqref{eq:fft_pde}, $\frac{\partial}{\partial t}\hat{h}$ will negatively response to $\hat{h}$, which means the perturbation will be damped with time.

% \section{Conclusion}

% In conclusion, we use FFT method to solve the PDE for the evolution of teh surface of a stressed solid. 
% We use Julia to implement the code for numerical solution and overcome the gap between the theorem and code.
% Then we simulate the evolution of the surface for different $r$ and find that the planar surface is unstable for $r=1.5,3.8$, but stable for $r=5.0$.
% Finally, we analyze the elementary reason for the stability and instability of the planar surface.



\bibliographystyle{splncs04}
\bibliography{refs}

\section*{Appendix}
\subsection*{Code}

\lstinputlisting[caption={EDR Computation},captionpos=t, label={EDR Computation}, language=Python]{../code/EDRCompute.py}

\end{document}

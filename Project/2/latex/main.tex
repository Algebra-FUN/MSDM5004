\documentclass[runningheads]{llncs}

\usepackage{graphicx}
\usepackage[utf8]{inputenc}
\usepackage{subcaption}
\captionsetup{compatibility=false}
\usepackage{amsmath}
\usepackage{amssymb}
\usepackage[colorlinks=true, linkcolor=blue]{hyperref}
\usepackage{tabularx}
\newcolumntype{C}{>{\centering\arraybackslash}X}
\usepackage{listings}
\renewcommand\lstlistingname{Code}
\usepackage[usenames,dvipsnames]{xcolor}

\definecolor{backcolour}{rgb}{0.95,0.95,0.92}
\definecolor{codegreen}{rgb}{0,0.6,0}
\definecolor{codepurple}{rgb}{0.58,0,0.82}
\lstdefinestyle{pyStyle}{
    backgroundcolor=\color{backcolour},   
    commentstyle=\color{codegreen},
    keywordstyle=\color{magenta},
    stringstyle=\color{codepurple},
    basicstyle=\ttfamily\footnotesize,
    breakatwhitespace=false,         
    breaklines=true,                 
    keepspaces=true,                 
    numbers=left,       
    numbersep=5pt,                  
    showspaces=false,                
    showstringspaces=false,
    showtabs=false,                  
    tabsize=2,
}
\lstset{style=pyStyle, language=Python}

\begin{document}

\title{
    MSDM5004 Project II\\
    Calculation of a Turbulence Metric by Spectral Methods on Wind Data}
\titlerunning{MSDM5004 Project II}
\author{FAN Yifei}
\authorrunning{Y. FAN}
\institute{HKUST, Hong Kong SAR, China\\
    \email{\href{mailto:yfanba@ust.hk}{yfanba@ust.hk}}}
\maketitle

\begin{abstract}
    In this project, we will perform spectral analysis on one QAR dataset to compute the EDR using a wind-based algorithm, 
    and compare the results with that provided by a commercial software employing the NRL algorithm. 
    Then We will apply the same methods on an AIMMS20 dataset to calculate the EDR, analyze the result and report the turbulence encountered by the aircraft.\\
    The code of this project is available at \url{https://github.com/Algebra-FUN/MSDM5004/tree/main/Project/2}.
    \keywords{Spectral Analysis \and Turbulence \and EDR}
\end{abstract}

\section{Introduction}

Atmospheric turbulence is a weather phenomenon which is a main cause of reported weather-related aircraft incidents and accidents. The impact of turbulence depends on magnitude and size of eddies in the air as well as other aircraft type specific parameters.
Among the wide range of eddy sizes in the atmosphere, aircraft mainly respond to eddies having sizes in the order of tens to hundreds of meters. Eddies of this size can be created by an energy cascade, i.e., a larger eddy in the atmosphere breaking up and forming a smaller eddy, which eventually dissipating to heat. Because this size range is outside the current scope of state-of-the-art operational Numerical Weather Prediction models, these turbulence forecasts are often still diagnostic based through parameterization, i.e., parameters used in algorithms producing these forecasts often need to be “calibrated” using turbulence observations.

One standard metric to quantify turbulence is the “cubic root of eddy dissipation rate” (EDR). 
EDR can be estimated by wind-based algorithm or acceleration-based algorithm, the current project focus on the wind-based method.

In this project, we will perform spectral analysis on one QAR dataset to compute the EDR using a wind-based algorithm, and compare the results with that provided by a commercial software employing the NRL algorithm. 
We will then apply the same methods on an AIMMS20 dataset to calculate the EDR.

\section{Turbulence Theory}

According to turbulence theory, frequency spectra of transverse velocity (vertical wind component being one) in the inertial sub-range of turbulence regime is given by:

\begin{equation}
    S_{\perp}(f)=C_k^{'}\left(\frac{U}{2\pi}\right)^{\frac23}\epsilon^{\frac23}f^{-\frac53},
\end{equation}
where $f$ is the frequency; $U$ is the average true air speed within the sampling window; $\epsilon$ the dissipation and $C^{'}_k$ being the Kolmogorov constant for transverse
component which is around $0.65$. From this equation, we can find the relation between power spectrum of the vertical wind component and the frequency follows a $-\frac53$ power law.

\section{Data Exploration and Verification(Task 1)}

In this section, we will explore the QAR dataset and verify the $-\frac53$ power law by fitting a linear relation to the log-log plot of the power spectrum of the vertical wind component in QAR dataset.

First, we load the QAR dataset and calculate the fourier transform of the vertical wind component. 
Then we calculate the power spectrum of the vertical wind component 
and plot the log-log plot of the power spectrum of the vertical wind component in Fig.~\ref{fig:qar_loglog},
where we can see the $\ln{S}$ is proportional to $\ln{f}$ and the line is quite noisy.

\begin{figure}[!htbp]
    \centering
    \includegraphics[width=\textwidth]{../img/qar_loglog.pdf}
    \caption{Log-log plot of the power spectrum of the vertical wind component in QAR dataset}
    \label{fig:qar_loglog}
\end{figure}

In order to identify the frequency range, bounded by $\omega_1$ and $\omega_2$ that best demonstrates the $-\frac53$ power law,
we should experiment with different sampling window on the vertical wind component to smooth the line. 
We experiment with different sampling window, varying from 10 second to 2 minutes. The result is shown in Fig.~\ref{fig:1}.
We show the fitted line coefficient and fitted goodness metric $R^2$ in the plot.
\begin{figure}[!htbp]
    \centering
    \begin{subfigure}{.4\textwidth}
        \centering
        \includegraphics[width=\linewidth]{../img/qar_loglog(w=10).pdf}  
    \end{subfigure}
    \begin{subfigure}{.4\textwidth}
        \centering
        \includegraphics[width=\linewidth]{../img/qar_loglog(w=20).pdf}  
    \end{subfigure}
    \begin{subfigure}{.4\textwidth}
        \centering
        \includegraphics[width=\linewidth]{../img/qar_loglog(w=50).pdf}  
    \end{subfigure}
    \begin{subfigure}{.4\textwidth}
        \centering
        \includegraphics[width=\linewidth]{../img/qar_loglog(w=120).pdf}  
    \end{subfigure}
    \caption{Log-log plot of the power spectrum of the vertical wind component in QAR dataset with different sampling window}
    \label{fig:1}
\end{figure}

From Fig.~\ref{fig:1}, the line between $0.1$ and $1$ is the smoothest and most straight part in the curve.
Then we choose $\omega_1=0.1$ and $\omega_2=1$ in following calculation.

\section{Calculation of EDR for QAR Dataset}

In this section, we will calculate the EDR for QAR dataset using ML method and NLR method.

\subsection{ML Method(Task 2)}

The Eddy Dissipation Rate (EDR) can be estimated by the Maximum Likelihood method with the following equation:

\begin{equation}
    \mbox{EDR}=\epsilon^{\frac13}=\left(\frac{2\pi}{U}\right)^{\frac13}\left[\frac1N \sum_{f=\omega_1}^{\omega_2}\left(\frac{S(f)f^{\frac53}}{C^{'}_k}\right)\right]^{\frac12},
\end{equation}
where $\omega_1$ and $\omega_2$ is the cut-off frequency, $S(f)$ is the power spectrum of the vertical wind component, $U$ is the average true air speed within the sampling window, $N$ is the number of data points in the sampling window, and $C^{'}_k$ is the Kolmogorov constant for transverse component which is around $0.65$.

\paragraph{Implementation}

We implement the ML method in Python. There are some details in the implementation:

\begin{itemize}
    \item The fft function in Numpy\cite{fft} is different from the theoretical definition of Fourier transform, the fft function in numpy normalizes on backward transform defaultly. 
    So we should set the parameter "norm" to "ortho" to get the right spectrum.
    \item We also can use rfft and irfft in numpy to calculate the Fourier transform and inverse Fourier transform of real sequence, which is faster than fft and ifft.
\end{itemize}

\paragraph{Experiment}

We carry out experiment with different sampling window $w \in [3,5,10,20]$, the result is shown in Fig.~\ref{fig:2}.

\begin{figure}[!htbp]
    \centering
    \begin{subfigure}{.4\textwidth}
        \centering
        \includegraphics[width=\linewidth]{../img/qar_edr(w=5).pdf}
    \end{subfigure}
    \begin{subfigure}{.4\textwidth}
        \centering
        \includegraphics[width=\linewidth]{../img/qar_edr(w=10).pdf}
    \end{subfigure}
    \begin{subfigure}{.4\textwidth}
        \centering
        \includegraphics[width=\linewidth]{../img/qar_edr(w=20).pdf}
    \end{subfigure}
    \caption{Estimation EDR of QAR dataset using ML method with different sampling window}
    \label{fig:2}
\end{figure}

From Fig.~\ref{fig:2}, when the sampling window $w=5$, the estimated EDR $\hat{\epsilon}^{\frac13}$ achieves the best MSE loss.

\subsection{NLR Method(Task 3)}

The Eddy Dissipation Rate (EDR) can be estimated by the NLR algorithm with the following equation:

\begin{equation}
    \epsilon^{\frac13}=\frac{\widehat{\sigma_w}}{\sqrt{1.05V_a^{\frac23}\left(\omega_1^{-\frac23}-\omega_2^{-\frac23}\right)}}
\end{equation}
where the $\widehat{\sigma_w}$ is the running standard deviation of the band-pass-filtered vertical wind component with cut-off frequency $\omega_1$ and $\omega_2$, 
$V_a$ is the low-pass-filtered airspeed. In this experiment, we use the airspeed within sampling window as $V_a$.

\paragraph{Experiment}

We carry out experiment with different sampling window $w \in [5,10,20]$, the result is shown in Fig.~\ref{fig:3}.

\begin{figure}[!htbp]
    \centering
    \begin{subfigure}{.4\textwidth}
        \centering
        \includegraphics[width=\linewidth]{../img/qar_edr_nlr(w=5).pdf}
    \end{subfigure}
    \begin{subfigure}{.4\textwidth}
        \centering
        \includegraphics[width=\linewidth]{../img/qar_edr_nlr(w=10).pdf}
    \end{subfigure}
    \begin{subfigure}{.4\textwidth}
        \centering
        \includegraphics[width=\linewidth]{../img/qar_edr_nlr(w=20).pdf}
    \end{subfigure}
    \caption{Estimation EDR of QAR dataset using NLR algorithm with different sampling window}
    \label{fig:3}
\end{figure}

From Fig.~\ref{fig:3}, when the sampling window $w=10$, the estimated EDR $\hat{\epsilon}^{\frac13}$ achieves the best MSE loss.

\section{Calculation of EDR for AIMMS20 Dataset(Task 4)}

In this section, we will calculate the EDR for AIMMS20 dataset using ML method and NLR method.

First of all, we load the data and plot the power spectrum of the vertical wind component in Fig.~\ref{fig:4}.

\begin{figure}[!htbp]
    \centering
    \includegraphics[width=.6\linewidth]{../img/probe_loglog.pdf}
    \caption{Log-log plot of the power spectrum of the vertical wind component in AIMMS20 dataset}
    \label{fig:4}
\end{figure}

We choose $\omega_1=0.1$ and $\omega_2=1$ in following calculation.

First, we calculate the EDR using ML method and NLR algorithm with different sampling window $w=5,10,20$ and 
vertical windspeed component, the result is shown in Fig.~\ref{fig:6}.

\begin{figure}[!htbp]
    \centering
    \begin{subfigure}{.4\textwidth}
        \centering
        \includegraphics[width=\linewidth]{../img/probe_edr(w=3).pdf}
        \caption{$w=3$}
        \label{fig:6a}
    \end{subfigure}
    \begin{subfigure}{.4\textwidth}
        \centering
        \includegraphics[width=\linewidth]{../img/probe_edr(w=5).pdf}
        \caption{$w=5$}
    \end{subfigure}
    \begin{subfigure}{.4\textwidth}
        \centering
        \includegraphics[width=\linewidth]{../img/probe_edr(w=10).pdf}
        \caption{$w=10$}
    \end{subfigure}
    \begin{subfigure}{.4\textwidth}
        \centering
        \includegraphics[width=\linewidth]{../img/probe_edr(w=20).pdf}
        \caption{$w=20$}
    \end{subfigure}
    \caption{Estimation EDR of AIMMS20 dataset using ML method with different sampling window and vertical windspeed component}
    \label{fig:6}
\end{figure}

\subsection{Turbulence Report(Task 5)}

From the Fig.~\ref{fig:6a}, we can see, at the end of flight, the estimated EDR is larger than $0.2$, 
which means the turbulence at that time should be reported as “Moderate”.
Although which only appears when $w=3$, we still report the turbulence at the end of flight as “Moderate”, since we should ensure the safety of flight and consider the uncertainty of estimation.

\subsection{Discussion on results}

We admit that the accuracy of EDR estimation is not enough, here are the reasons:

\begin{enumerate}
    \item Only consider the vertical wind component, other wind components may contribute to the turbulence.
    \item The frequency of the data is not high enough (only 1HZ) which may cause the loss of information on high frequency. So you may collect more data with higher frequency to improve the accuracy.
\end{enumerate}

% Furthemore, in order to get a more precise estimation of EDR, we calculate transverse velocity instead of only vertical velocity.

% The transverse velocity is calculated by the following equation:

% \begin{equation}
%     w_{\perp}=\sqrt{|\vec{w}|^2-\left(\frac{\vec{w}\cdot \vec{v}}{|\vec{v}|}\right)}
% \end{equation}

% where $v\in\mathbb{R}$ is the aircraft velocity vector and $w\in\mathbb{R}$ is the wind velocity vector.

% We compute EDR using ML method and NLR algorithm with different sampling window $w=5,10,20$ and transverse velocity, the result is shown in Fig.~\ref{fig:5}.

% \begin{figure}[!htbp]
%     \centering
%     \begin{subfigure}{.4\textwidth}
%         \centering
%         \includegraphics[width=\linewidth]{../img/probe_edr_t(w=5).pdf}
%     \end{subfigure}
%     \begin{subfigure}{.4\textwidth}
%         \centering
%         \includegraphics[width=\linewidth]{../img/probe_edr_t(w=10).pdf}
%     \end{subfigure}
%     \begin{subfigure}{.4\textwidth}
%         \centering
%         \includegraphics[width=\linewidth]{../img/probe_edr_t(w=20).pdf}
%     \end{subfigure}
%     \caption{Estimation EDR of AIMMS20 dataset using ML method with different sampling window}
%     \label{fig:5}
% \end{figure}

\section{Conclusion}

In this conclusion, we performs spectral analysis on one QAR dataset to compute the EDR using a wind-based algorithm, and compare the results with that provided by a commercial software employing the NRL algorithm. 
Our prediction is relative accurate and the MSE loss is small enough, considering that we only use the vertical wind component and the frequency of the data is not high enough.

Then we apply the same methods on an AIMMS20 dataset to calculate the EDR, analyze the result and report the turbulence encountered by the aircraft. From the result, it seems there is no turbulence during the flight, but we still report the turbulence at the end of flight as “Moderate”, since we should ensure the safety of flight and consider the uncertainty of estimation.

In further study, we should include other wind components and collect more data with higher frequency to improve the accuracy of EDR estimation.

\begin{thebibliography}{8}
    \bibitem{fft}
    umpy: numpy.fft, \url{https://numpy.org/doc/stable/reference/generated/numpy.fft.fft.html}, last accessed: May 4 2023
    
    \bibitem{citekey}
    Lee JCW, Leung CYY, Kok MH, Chan PW. A Comparison Study of EDR Estimates from the NLR and NCAR Algorithms. Atmosphere. 2022; 13(1):132. 

    \bibitem{citekey}
    Wacławczyk M, Gozingan AS, Nzotungishaka J, Mohammadi M, P. Malinowski S. Comparison of Different Techniques to Calculate Properties of Atmospheric Turbulence from Low-Resolution Data. Atmosphere. 2020; 11(2):199.
\end{thebibliography}

\section*{Appendix}
\subsection*{Code}

\lstinputlisting[caption={EDR Computation}, captionpos=t, label={EDR Computation}, language=Python]{../code/EDRCompute.py}

\end{document}
